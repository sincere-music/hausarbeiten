\section{Einleitung}

Ein großer Wurf ist Ludwig van Beethoven’s siebtes Streichquartett. Nicht
nur wendet er sich damit nach Jahren erstmals wieder dieser Gattung zu\footnote{
sieht man von kontrapunktischen Studien ab, auf die Kramer \parencite{kramer}
hingewiesen hat} und überträgt somit erstmals seinen „neuen Weg“\cite{dahlhaus}
auf das Streichquartett.  Auch eröffnet er damit ein Opus aus dreien, in dem
vieles, beispielsweise der sehr ausführliche Beschluss des dritten Quartetts,
auf eine zusammenhängende Konzeption weist.  Herausragend sind am ersten Quartett
allein die äußeren Dimensionen mit um die 40~Minuten Spieldauer, sowie die
virtuose Behandlung des Streichquartetts und die Anforderungen an die
Ausführenden, technischer ebenso wie interpretatorischer Natur. —

Das allein, verbunden mit der kompositorischen Qualität eines Beethoven
in Bestform, würde genügen, dem Werk einige Aufmerksamkeit zu sichern;
dazu gesellen sich aber gewisse Widersprüche: Einerseits ist offenbar,
dass die mittleren Quartette weit mehr auf Öffentlichkeitswirksamkeit
berechnet sind als die frühen des op.\,18. – zugleich aber gibt die Formen-~und
Tonsprache bereits eine Vorahnung der späteren Wendung ins Hermetische.
Während die Modernität des zweiten Satzes in äußerster Raffinesse der
Form sowie einer irrlichternden, leichtfüßigen Stimmung liegt, erleben
wir im dritten Satz formal eine fast rückwärtsgewandte Schlichtheit bei
zutiefst betrübtem Affekt und ungebrochener Expressivität.

Spätestens durch diese Überraschungen regt das Werk seit seiner Entstehung
ein ungebrochenes Interesse an, das wohl noch größer ist als das für die
anderen Quartette dieser \enquote{mittleren Periode}, wie eine große Fülle
an theoretischen Deutungen ebenso wie die Stellung des Werkes im Kanon der
Streichquartett-Literatur und in den Konzertprogrammen unter Beweis
stellt.  Wozu nun noch eine weitere Interpretation des Stückes verfassen,
ist denn noch nicht alles gesagt, was Worte dazu sagen können?

Vor allem zwei Aspekte geben hierzu Anlass: zum Einen besteht die Hoffnung,
dass in der großen Fülle und Dichte der Ideen wir noch Neues entdecken
können.  Zum Anderen scheint uns eine Balance noch nicht recht gefunden
zwischen der mehr spekulativ-empathischen Deutungsweise, die wir bis zum
zweiten Weltkrieg in den verschiedensten Ausprägungen, z.\,B. bei Theodor
Helm\cite{helm} und Arnold Schering\cite{schering}, finden, und den eher
analytisch-rationalen Deutungen der Zeit danach.  So wollen wir uns daran
wagen, indem wir zunächst einen detaillierten Gang durch die einzelnen
Sätze gehen und schließlich aufgrund der gewonnenen Erkenntnisse einen
tieferen Einblick in Dramaturgie und Charakter des Quartetts zu gewinnen suchen.