\section{I. Allegro}

Beethoven eröffnet sein zweites großes Streichquartett-Opus, indem
er viele Erwartungen enttäuscht.  Kein prägnantes Motiv steht am
Anfang, nicht einmal ein Grundton, geschweige denn ein vollständiger
Akkord, und die erste Geige schweigt für die ersten acht Takte. – Der
lange Atem des Hauptthemas\footnote{welches darin den Hauptthemen
der benachbarten Konzerte op. 58 \& 61 ähnelt} wird auch durch die
sehr großflächige Harmonisierung herausgestrichen: der erste Harmoniewechsel
in Takt 7 findet sehr unauffällig in der Mitte des Taktes statt, und danach
wechselt die Harmonie erst wieder mit Takt 19.
\begin{notenbeispiel}
 \includegraphics{nbs/1-motivik.pdf}
 \caption{Motive im Hauptthema des ersten Satzes}
\end{notenbeispiel}


Darüberhinaus ist die weitere Entwicklung ungewöhnlich: den großen
Raum, den bereits die Exposition einnimmt, füllt Beethoven mit vielen
recht lose verknüpften Abschnitten.  Der ersten \enquote{Idee} des Hauptsatzes
folgt eine zweite (T.~19) und eine dritte (T.~30), bevor in T.~38
erstmals aus Motiv $a$ eine typische modulierende
Überleitung entsteht.  Das Cello-Solo markiert nach dem \lilyDynamics{ff}-Akkord
von T.~48 einen großen dramaturgischen Einschnitt, dem aber nicht –
wie zu erwarten – der Seitensatz folgt, sondern eine weitere Idee,
die mit harschen Dissonanzen, verminderten und übermäßigen Intervallen,
chromatischer Bassführung sowie \lilyDynamics{sfp}-Stichen einen völlig
anderen Affekt ins Spiel bringt.  Mit dem \tonart{C-Dur} in T.~58 löst
sich diese Episode in Wohlgefallen auf und jetzt folgt das eigentliche
Seitenthema.

An sich besitzt dieses eine ganz regelmäßige viertaktige Periodik, die
Beethoven jedoch durch die Bogensetzung über den vierten und fünften Takt
sowie die Phrasenverschränkung (\enquote{Takterstickung}) in T.~67 sogleich
zu weiten sucht.  T.~73 bringt die zweite Idee des Seitensatzes, eine
Triolen-Rhythmik und -Motivik, die im Verlauf des Satzes breiteren Raum
einnehmen wird (und bereits in T.~42\,ff. vorbereitet wurde).  Mit T.~85
folgt ein weiterer Bruch mit geisterhaften, alleinstehenden Halbe-Akkorden,
deren Bindeglied \tonart{d-moll} erst in T.~90 nachgeliefert, und direkt
in eine typische Schlussgruppe weitergeführt wird.

Vorgetäuscht wird nun eine Wiederholung der Exposition mit dem Hauptthema
in der Haupttonart, das jedoch bereits nach fünf Takten aus seinem
harmonischen und dynamischen Fluss ausbricht und nach dreitaktigem
Verweilen auf dem verminderten Akkord ges-c-es die Durchführung in \tonart{B-Dur}
eröffnet.  Die unschlüssige Harmonik des Hauptthemas manifestiert sich
hier in einer gegenseitigen Verschiebung der viertaktigen Harmonik gegen 
die viertaktige Melodik.  Wie öfters in diesem Satz wird eine Verunsicherung
des analytischen Rezipienten von einem eindeutig erkennbaren Formteil
abgelöst: kontrapunktische Verdichtung, Motivabspaltung ($b$ in T.~124\,ff.)
und ein längerer Aufenthalt in \tonart{d-moll} (Tonikaparallele als traditionelle
Haupttonart der Durchführung) sind geradezu exemplarisch
eingesetzt.  Konsequenterweise verlässt Beethoven jedoch die klassischen
Wege direkt wieder, indem nach T.~147 nicht etwa die Rückführung in die
Reprise folgt, sondern sich zwei weitere Durchführungsabschnitte anschließen.

Der zweite Durchführungsabschnitt in T.~152–184 ist eng an das Hauptthema
angelehnt.  Motiv $b$ wird nun von der ersten Geige über ruhenden Akkorden
fortgesponnen und die Musik wendet sich in einer sich beschleunigenden
Sequenz nach \tonart{Des-Dur}, welches wie zuvor \tonart{d-moll} über eine
lange Strecke (T.~160–183) bestätigt wird.  Abrupt reißt uns T.~184 aus dem 
\enquote{Idyll}\cite[S.~150]{indorf}; die Kluft wird zwar durch eine
Sequenzierung des Motivs $a'$ überbrückt, jenseits ihrer befinden wir uns
aber mit einer ausgedehnten Reminiszenz an barockes Fugenidiom in einer
völlig anderen stilistischen Sphäre.

Was die zweite Geige als Kontrapunkt (oder zweites Thema) beigibt, sind
in der Tat barocke Allerweltsmotive ohne große Individualität\footnote{Wir
wollen allerdings die Möglichkeit zulassen, eine Beziehung zu Motiv $c$
zu sehen.}: wie wir bei Richard Kramer\cite[S.~239ff. und 248]{kramer}
nachlesen, hatte die erste Note ursprünglich noch einen Quartauftakt
(analog zum Sextsprung des dritten Takts), der in der Endfassung
gestrichen wurde.  Der Eindruck der Reminiszenz speist sich aus Aspekten
wie der \lilyDynamics{pp}-Dynamik, dem ebenfalls nur zweistimmig ausgesetzten zweiten
Einsatz (eine gewisse anfängliche Ziellosigkeit spricht hieraus) und
der gewissen (beabsichtigten?) Ungeschicklichkeit in der Handhabung
des barocken Idioms (Kramer weist auf die ungewöhnliche Form der
Themenbeantwortung hin).  Plötzlich bringt T.~201 doch eine Dramatisierung
der Episode: \textit{[molto] crescendo} steigt die Musik eine weitere Quinte von
b-~nach \tonart{f-moll}; eine Art Engführung in harschen Dissonanzen treibt noch
weiter nach \tonart{c-moll}, wo sich die aufgetürmte Spannung entlädt und nach
einer Synkopenkette auf einem \enquote{leeren} G zur Ruhe kommt.

Dies wäre der richtige dramaturgische Einschnitt, um darauf die Reprise
folgen zu lassen.  Aber wir befinden uns ein bis zwei Quinten zu hoch,
und wenngleich Gestus und Motivik des Beginns nach der fremdartigen
Episode \enquote{im alten Stil} wiederkehren, wird doch weiter moduliert,
und Kombinationen von Motiv $a$ mit der Triolenmotivik wandern durch
die Stimmen, ohne dass sich eine Stringenz ergibt.  Selbst die
steigenden Linien in T.~236f. bleiben durch die
\textit{sempre}~\lilyDynamics{p}-Vorschrift gänzlich undramatisch,
bevor die erste Geige aus ihrem Schweben in höchster Höhe durch das
lang erwartetete \tonart{F-Dur} im \lilyDynamics{f} herabgeholt wird: ein weiterer
Schritt in Richtung Reprise ist genommen.  Doch es handelt sich um die
\emph{zweite} Idee des Hauptsatzes, die hier wiederkommt, und so geht
Beethoven’s Spiel noch weiter: T.~252 bringt \lilyDynamics{ff} die
Terz \mbox{c-a} wieder, mit der das Stück eröffnet wurde, und nochmals
zwei Takte später ist die gesamte Konstellation des Anfangs wieder
da und räumt beim Zuhörer alle Zweifel aus.  Indes,
eine letzte Verklammerung von Reprise und Durchführung lässt sich
Beethoven noch einfallen: Nachdem das Hauptthema ab T.~266 nach
\tonart{f-moll} abgebogen war, kehrt das \enquote{Hornmotiv} der Exposition in Des-Dur
wieder und so entsteht eine deutliche Beziehung zum zweiten Teil
der Durchführung, der sich so lange in Des-Dur aufgehalten hatte.
Anders als in der Exposition ist auch, dass die folgende Überleitung
nicht aus dem Hauptthema, sondern als Fortspinnung aus der \enquote{Hörnerstelle}
gewonnen wird.  Schließlich kehrt in T.~295 das Baß-Solo
von T.~48 wieder und ab diesem Punkt werden die weiteren Stationen der
Exposition getreu in die Haupttonart versetzt.

Zu guter Letzt erhält dieser weit ausladende Satz noch eine Coda.  Sie
beginnt in T.~348 mit einer Affirmation, die bisher keine Parallele hat: Das
Hauptthema wird im \lilyDynamics{ff}, in einer ebenso prächtigen
wie detailreichen Harmonisierung (denkbar starker Kontrast zur Klangfläche
der ersten Takte) und voller Ausschöpfung der Quartett-Klanglichkeit
präsentiert.  Wie erschöpft von dieser Anstrengung, wird die in T.~357
zunächst offen gelassene Phrase lyrisch zu Ende geführt.  –  Erneuter
Auftritt der Triolenmotivik; eine weitere Kombination mit einem
Quasi-Kanon, gewonnen aus Motiv $a$; und ein weiterer großer Aufstieg
zum c\textsuperscript{4}, das in diesem Satz so markant Verwendung
gefunden hat und jetzt über mehr als fünf Takte ausgehalten wird.  Nach
einer weiteren großen Beruhigung und einer letzten Überraschung mit dem
\lilyDynamics{pp} \tonart{d-moll} T.~396 schließt der Satz mit zwei großen
\lilyDynamics{ff}-Akkorden.