\section{II. Allegretto vivace e sempre scherzando}

\subsection{Das zugrundeliegende Formmodell}
Wo im ersten Satz Enttäuschung und Erfüllung formaler Erwartungen
einen recht regelmäßigen Wechsel hielten, da spielt der zweite
Satz an der Oberfläche fast durchgehend ein formales Verwirrspiel.  Der
paukenartige Rhythmus, den das Cello zu Beginn exponiert, und die
verspielte Antwort der zweiten Geige durchziehen den gesamten Satz
in rascher Folge, und die tonartliche Konzeption ist zunächst
undurchschaubar, da sich schnelle Modulationen und unvermittelte
Rückungen wie bereits die zu T.~9 mit häufigem Zurückfallen nach
\tonart{B-Dur} (T.~1, 29, 68, 101) mischen.  Suchen wir also als
Erstes nach eindeutigen Anhaltspunkten in diesem planvollen
Durcheinander:

Zweierlei Episoden sind durch ihren Gestus klar von der
Hauptthematik dieses Satzes unterschieden und treten auch nach
sehr klaren Zäsuren auf.  Die erste in d-moll erstreckt sich von
T.~39–61, die zweite in \tonart{f-moll} umfasst die T.~115–150.  Beide
werden ab T.~275 resp. 354 quasi unverändert wiederaufgenommen,
und dass diese Wiederaufnahme jeweils eine Quinte tiefer erfolgt,
veranlasst uns zur Hypothese, dass \emph{beide} Seitenthemen einer
Sonatensatzform sind.

Diesem Hinweis lässt sich nachgehen, und in der Tat lassen sich dann
weitere Stationen eines Sonatensatzes zwanglos verorten: Es wäre in
T.~239 eine tonal verschleierte Reprise anzusetzen, welche Deutung
nach unserer Meinung durch die Analogie zum ersten (und vierten) Satz sehr an
Plausibilität gewinnt.  Ein weiterer Parallelismus zwischen T.~151
und 390 trägt zur Identifikation der beginnenden Durchführung und
Coda bei, da es sich erkennbar um eine Variante des ersten Themas
handelt ($151\widehat{=}1$, $155\widehat{=}5$).  Wirklich erstaunlich
ist nun, dass diese vier Formteile mit \mbox{150:88:151:87} sehr regelmäßig
proportioniert sind: die Generalpause vor dem Einsatz der Reprise in
239 teilt den Satz in zwei exakt gleich lange Hälften!

Bevor wir nun die Entfaltung dieses Formmodells im Detail betrachten,
ist noch ein Einwand zu entkräften: Die zweimalige Rückkehr nach \tonart{B-Dur}
in den T.~68 und, noch emphatischer, 101 scheint kaum in einen Sonatensatz
zu passen, schon gar nicht in der prozesshaften Ausprägung, die sich
bei Beethoven in dieser Zeit beobachten lässt.  Nancy November formuliert:
\foreignquote{english}{This tonal return makes sense if one
hears the movement as a scherzo and the new material introduced in bar~115
as a first trio, as do Kerman and Del Mar.}\cite[p.~65]{november}

An ein Scherzo-Formmodell zu denken ist natürlich durch die
Satzüberschrift nahegelegt.  Der Vergleich mit Werken verschiedener
Schaffensperioden zeigt aber, dass Scherzi bei Beethoven nie
Spielfeld tiefgreifender formaler Experimente waren, sondern im
Gegenteil durch formale Einfachheit profiliert:  Scherzo und Trio
sind immer deutlich voneinander getrennt, in der Regel auch tempomäßig
unterschieden, ganze Formteile werden wörtlich wiederholt, z.T.
unter Änderung eines einzelnen Parameters wie der Dynamik.  Davon
sind wir mit dem komplexen formalen Geschehen in diesem \enquote{Allegretto
vivace sempre scherzando} weite entfernt, sodass wir eher geneigt
sind, eine klare Unterscheidung zwischen \enquote{Scherzo} als
Satzüberschrift\footnote{die Beethoven allerdings nicht immer
explizit setzt} und \textit{sempre scherzando} als charakterliche
Spezifizierung des Allegretto\footnotemark, analog zum
\textit{mesto} des dritten Satzes, zu beobachten.
\footnotetext{Beethoven verwendet Allegretto oft in Sätzen, die
man als \enquote{langsame Sätze} klassifizieren würde, ein Begriff, der
für das Kontinuum zwischen (Molto) Adagio – Andante – Andantino
quasi Allegretto – Allegretto \&c.\ schlecht passen will.  Es
scheint, dass ein treffenderer nicht erfunden ist, weder ein Oberbegriff
für die Mittelsätze einer symphonischen Form, die bei Beethoven ja
in verschiedenster Form und Reihung auftreten, noch ein Begriff
für diesen Sonderfall (gibt es überhaupt einen vergleichbaren?).}

Zwei weitere Aspekte können die tatsächliche Funktion dieser
Rückwendung nach \tonart{B-Dur} erhellen:  Zunächst offenbart
ein genauerer Blick auf die Motivik von T.~39\,ff. eine enge
Verwandtschaft zum spielerischen Geigeneinwurf des Anfangs
(T.~5\,ff.) (siehe Notenbeispiel \vref{nbs:2-var}).
\begin{notenbeispiel}
 \includegraphics{nbs/2-var-d-moll.pdf}
 \caption{Motivvergleich in der ersten Themengruppe}
 \label{nbs:2-var}
\end{notenbeispiel}

Außerdem bringt wie so oft
der Blick in die Reprise weitere Klarheit über die Verhältnisse
in der Exposition: Zwar wird das \tonart{d-moll}-Thema in der Reprise wie
ein zweites Thema quinttransponiert und die erste \enquote{Rückkehr} der
Exposition führt hier nicht nach B- sondern \tonart{F-Dur}, aber die folgenden
Modulationen weichen raffiniert davon ab (T.~327\,ff. bewegen sich
wie ihre Vorgänger aus der Exposition in einer \tonart{d-moll}-Tonalität,
aber innerhalb dessen auf der V.~Stufe) und in 337 schließt sich
auch dies als Bogen nach \tonart{B-Dur}, mit noch gesteigerter
Emphase (Melodie in drei Oktaven parallel geführt!).

Eine erste Themengruppe von 114 Takten Länge, die in sich eine
Bogenform darstellt und erst in den letzten (circa) acht Takten
tatsächlich zum zweiten Thema überleitet – das sind die
formidablen Dimensionen, die Beethoven in diesem Quartett eröffnet.

\subsection{Der Formverlauf im Detail}
Die außerordentliche Raffinesse lohnt noch einen detaillierteren
Blick auf das Geschehen dieses Satzes.

Regelmäßige Periodik und angenehm spielerische Rhythmik – darin
beginnt der Satz harmlos.  Ansonsten bekommt der Hörer aber wenig,
woran er sich halten kann: Vier solistische Einwürfe, \lilyDynamics{pp},
registermäßig weit voneinander entfernt, hängen in der Luft.  Nach
dem Halbschluss T.~8 folgt eine Rückung des Gehörten sechs Töne nach
oben, die die harmonische Orientierung mutwillig verwirrt.  Erst
von dem \tonart{F-Dur}-Terzquart-Akkord in T.~21 aus merken wir, dass das
vorher repetierte \tonart{Ces-Dur} uns in subdominantischer Funktion
bereits auf den Rückweg gebracht hat, und mit einer lieblichen
legato-Melodie klingt es, als wäre nichts gewesen.\footnote{Im
Wechselspiel zwischen Paukenrhythmus und Antwort steht diese Melodie
anstelle des verspielten Geigenmotivs von vorher, und es will uns
scheinen, als sei eine gewisse Ähnlichkeit im melodischen Gestus
vorhanden, vielleicht zu verorten in der Folge \mbox{d-c-b} oder in der
Öffnung zum f.} Ein abgespaltenes \lilyDynamics{f}-Motiv, eine
\lilyDynamics{p} hingetupfte Dominante, damit ist dieser
phantastische erste \enquote{Satz} gekonnt gerundet, und das folgende
\lilyDynamics{ff} leitet bereits über.  Auf einmal spukt es nun (T.~35\,ff.)
aus dem ersten Satz herüber: T.~85 und ähnliche waren dort vollkommen
losgelöst von der sonstigen Entwicklung und erscheinen nun
fast als Vorwegnahme des irrlichternden Charakters, der den
zweiten Satz ausmacht – für eine solch assoziativ-ungeordnete
Kreativität, oder doch zumindest deren Niederschlag außerhalb von
Fantasien, ist Beethoven sonst nicht bekannt!

Die jetzt folgende d-moll-Melodie zeigt einen sehr reizvollen
Umgang mit dem Wechsel von regelmäßiger und unregelmäßiger Periodik:
Das Motiv von T.~5 hatte auftaktig begonnen, hier wird es durch den
\lilyDynamics{fp}-Akkord volltaktig ergänzt und ergibt so einen
zunächst fünftaktigen Vordersatz, mit zwei angehängten Takten, in denen
der rhythmische Impuls verebbt.  Hierauf folgen nun zwei achttaktige
\enquote{Nachsätze}; der erste besteht aus zwei parallelen Gliedern mit
Stimmentausch, der zweite nimmt die rhythmischen Keimzelle des Satzes
wieder auf und transferiert ihn in eine feierliche, fast funerale Stimmung.

Es folgt eine wiederum knappe Überleitung zurück nach \tonart{B-Dur}.  Von
einer Wiederholung des Anfangs kann nun keine Rede sein, denn die
Antwort auf den Rhythmus des Cello solo hat sich bereits gewandelt
in eine neue thematische Gestalt, die an prominenter Stelle wiederkehren
wird (T.~101, 213, 337).  Diese Variante wendet sich nun nicht nach
\tonart{As-Dur}, sondern nach \tonart{A-Dur}, von wo der Weg direkt nach
\tonart{d-moll} und zu einer hochdramatischen \lilyDynamics{ff}-Harmonisierung
des Anfangsrhythmus führt.  T.~99f. schließt einen weiteren Bogen
(vgl.~65\,ff.), bevor wir die erste Themengruppe und den \tonart{B-Dur}-Bereich
endgültig verlassen.

Im \tonart{f-moll}-Seitenthema arbeitet Beethoven mit einfachsten
Mitteln (Vorder- und Nachsatz sind fast identisch), sucht aber
genau wie beim Seitenthema des ersten Satzes die Weitung der kleinen
Einheiten mittels Takterstickung (T.~128 \& 141). Auch durch den Wechsel
zwischen Quinttransposition nach \tonart{c-moll} und Originallage
entsteht ein großer, empfindsamer Bogen.  Kontrapunktisch wird die
Wiederholung der T.~123–135 durch Stimmentausch neu beleuchtet,
bevor in T.149–151 das Motiv in immer tieferer Lage verebbt.

Überlappend beginnt Beethoven nun mit einer Variante des
Eingangsthemas, in durchgängig vierstimmigem Kleid; das erste
Glied wird harmonisch, das zweite kontrapunktisch und rhythmisch
sehr angereichert.  In typischer Durchführungsmanier wird das
rhythmische Modell der letzten beiden Takte weitergesponnen und
zur Modulation verwendet: zunächst ist der absteigende Bass recht
konventionell harmonisiert, dann wird G\textsuperscript{7} von
einer V. zur VI. Stufe umgedeutet und ein \lilyDynamics{ff}-Unisono-fis
wirft uns schockartig auf die andere Seite des Quintenzirkels.
Dort wartet wieder das \enquote{als wäre nichts gewesen}-Thema aus
T.~23, diesmal in \tonart{H-Dur}.

Weiter geht das Spiel mit dem Rhythmus, diesmal über einer Art
Ro\-ma\-nes\-ca-Bass\-mo\-dell, das zweimal in verschiedener Tonart folgt
und jeweils \textit{poco rit.} verebbt.  In C-Dur angekommen, beginnt
Beethoven nun einen Quintstieg, während dessen das Paukenmotiv
durch die Instrumente nach oben wandert, jeweils in Oktavgriffen
sowie unter idealer Nutzung der Klangfülle leerer Saiten.  Das
Verweilen in \tonart{a-moll} wird noch durch einen verminderten
Septakkord vorbereitet, dann folgt wie bereits erwähnt die
Motivvariante aus T.~72.  Sie wird nun im Dialog verdichtet, auf
ein Vierton-Sechzehntelmotiv reduziert und schließlich bereiten
diese rastlosen Sechzehntel aller vier Instrumente, \textit{sempre
staccato e piano}, in typischer Weise die Reprise vor.

Trugschlüssig setzt das Thema zwar in \tonart{Ges-Dur} statt B
wieder ein, aber in T.~259–270 wird der Hörer allmählich versichert,
dass er sich tatsächlich in der Reprise befindet\footnote{Man darf wohl
grundsätzlich voraussetzen, dass gebildete Hörer – und das
typische Streichquartett-Publikum war gebildet – um 1800 in der
Lage waren, eine Sonatensatzform hörend nachzuvollziehen; bei
einem Satz wie diesem ist das, gelinde gesagt, sehr fraglich.}.  Diese
Reprise verläuft analog zur Exposition und wir wollen daher nur
noch auf die neuerliche Variante des Beginns hinweisen, in der der
\enquote{Paukenrhythmus} durch eine neue gesangliche Linie in Sextenparallelen
komplementiert wird.  (Tatsächlich könnte man sagen, dass der
Eintritt der \emph{Reprise} als großer formaler Einschnitt paradoxerweise
gerade durch das Auftreten dieses \emph{neuen} Motivs markiert wird.)

Die Coda übernimmt zunächst wörtlich den Beginn der Durchführung,
lediglich eine Quinte nach unten transponiert.  Jedoch vor dem
enharmonischen Modulationsschritt, der in T.~168/69 stattfand,
beginnt das Cello ein Fugato in durchgehenden Sechzehnteln, das
fast durchweg nur ein Viertonmotiv sequenziert\footnote{und damit deutlich
an das Ende der Durchführung erinnert.}.  Nach einer chromatischen
Dramatisierung dieser Sequenzierung und einem Plateau in
F\textsuperscript{7} folgt die einzige harmonisch getreue
Wiederaufnahme des Beginns, begleitet vom lyrischen Motiv der
Reprise und in der \enquote{spielerischen Antwort} instrumentatorisch
raffiniert durch Fragmentierung und Verteilung der Sechzehntellinie.

Nun folgt ein weiteres Phänomen, das sich sicherlich nur bei
gründlichem formanalytischem Blick in die Partitur erschließt:
die enharmonische Modulation wird nachgeliefert und
landet auch genau da, wo sie nach T.~404 gelandet wäre –
in e.  Fugato und Reprise\footnote{Wir erinnern uns, dass
«\,Reprise\,» seinerzeit noch für jede Art von \enquote{Wiederholung},
weniger als formtheoretischer Spezialbegriff verwendet wurde.} des
Hauptthemas erscheinen als ein Einschub.  Nachdem die letzten
Inkarnationen des Themas etwas friedvolles, beruhigtes hatten,
wird der Hörer hier noch einmal arg überrascht: was vorher
ein \enquote{als wäre nichts gewesen} war, taucht hier in e-moll auf,
in tristem Tonfall, und es ist eine \enquote{Teufelsmühle},
die uns hier zurück nach B-Dur führt.  Nach acht Takten ist auch
dieser Spuk vorbei und heiter schließt der Satz, zunächst
entschwindend und dann mit kräftigen \lilyDynamics{ff}-Akkorden.