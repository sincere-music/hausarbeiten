\section{III. Adagio molto e mesto}

Wer sich vergleichbar raffinierte formale Strukturen auch
vom dritten Satz erhofft, wird enttäuscht.  Exemplarisch lässt
sich das bei Joseph Kerman beobachten, und wir sehen es
als symptomatisch für die Sachlichkeit des 20. Jahrhunderts
an, wenn er, dessen Buch über die Streichquartette
Beethoven’s ja als Standardwerk gilt, für diese Musik, ihren
starken, hochbetrübten Affekt und ihre relative formale
Schlichtheit jegliches Verständnis vermissen lässt\cite[S.~109\,ff.]{kerman}.  Dabei
zitiert Kerman selbst einen wesentlichen Schlüssel zum Verständnis: die
Überschrift \enquote{Einen Trauerweiden oder Akazien-Baum aufs Grab meines
Bruders}, die sich in einer der Skizzen findet. – Doch wir
greifen vor; zunächst der Blick in die Gestalt dieses Satzes.

In der Tat hat die Sonatensatzform hier eine ziemlich retrospektive
Ausprägung: die klare, beispielhafte Ausprägung der interpunktischen
Stationen in der Exposition\footnote{Gegenüber dem Archetyp
interpunktischer Form, den Koch in seiner Kompositionslehre formuliert
\parencite{koch}, sind lediglich die ersten zwei Stationen bereits
umgekehrt im Sinne einer geschlossenen Ausprägung des ersten Themas.}
würde man so bei Haydn zwanzig bis dreißig Jahre früher erwarten.  Im
Einzelnen sind es: Halbschluss in der Haupttonart T.~8; Ganzschluss
in der Haupttonart T.~16 (diese Zäsur vor der Überleitung noch
verstärkt durch eine Generalpause!); Halbschluss in der Nebentonart
T.~23; Ganzschluss in der Nebentonart T.~39, 41, 43 \&~45.

In seinem extrem langsamen Tempo entfaltet auch dieser Satz
enorme Dimensionen; gleich zu Beginn wird das achttaktige Thema
\emph{zweimal} vorgestellt, mit Stimmentausch und beim zweiten Mal
weiter angereichert durch die expressive Gegenstimme der ersten
Geige.  Nachdem das Thema zweimal \textit{morendo} verebbt ist,
hebt die Überleitung aus der Tiefe an und steigt imitatorisch
nach oben: durch die zweimalige Oberquintbeantwortung landen wir
sehr bald auf der Doppeldominante \tonart{G-Dur} und pendeln über
fünf Takte zunächst mit deren schmerzlich eingefärbter Dominante,
dann mit der neuen Tonika \tonart{c-moll}.

Der Seitensatz kann als eine siebentaktige Phrase angesehen werden,
die allerdings recht offen mit einem neuen rezitativischen Gestus
weitergeführt wird.  Diese neuerliche Überleitung verdichtet sich,
indem der Quintfall stark beschleunigt und ein Seufzermotiv dann
horizontal und vertikal immer dichter verknüpft wird, bis zum
dramatischen Höhepunkt in T.~36.  Auch die Schlussgruppe hat ihr
eigenes, filigran-repetitives Motiv, das im Dialog zwischen erster
Violine und Cello die vielen aufeinanderfolgenden Ganzschlüsse
einleitet.  Mit ihnen ist in T.~45 wieder fast jeglicher Impetus
zerflossen; ein registermäßig losgelöster \tonart{Es-Dur}-Terzquartakkord
bereitet denkbar schlicht den nächsten Abschnitt vor.

Endlich folgt nun die ersehnte Aufhellung: Die Durchführung
beginnt mit dem Seitenthema, das zunächst recht ausführlich in
\tonart{As-Dur} wiederkommt.  Der Modulationsweg führt dann
mit Motiven des Seitenthemas zunächst in ein feierlich-samtiges
\tonart{Des-Dur}, dann dramatisierend in ein strahlendes,
\lilyDynamics{ff} \tonart{D-Dur}; für \emph{einen} langen Takt ist
hier die Bewegung quasi aufgehoben, Kontrapunktik und 32tel-Begleitung
halten Pause.

Ein Wendepunkt zum zweiten Teil der Durchführung ist markiert
durch das \lilyDynamics{sf} in T.~56; wieder setzen durchgehende
32tel ein, ab T.~59 als generalbassmäßige Figuration des
\textit{pizzicato}-Cello.  Darüber werden nun Motive aus dem
ersten Thema entfaltet, in raffinierten Imitationen und
Sequenzmodellen\footnote{Großartig der Bogen, in dem der Bass
ab T.~61 Mitte langsam steigt und dann in T.~65f. rapide abfällt.}
verdichtet, und nach dem Höhepunkt sind wir auf ein Dominant-Plateau
gelangt, wo das Schlussgruppen-Motiv einen Abfall der Spannung
begleitet.

Es ist wirklich atemberaubend, wie Beethoven aus dem völligen
Schwebezustand in T.~70 (äußerst ausgedünnter Klang, harmonischer
Stillstand) im \textit{poco ritardando} nach \tonart{Des-Dur}\footnote{
welches nicht nur in dieser Durchführung bereits vorkam, sondern vor
allem im ersten Satz eine so prominente Rolle gespielt hat!}
moduliert und ein ganz neues gesangliches Thema einführt.  Adorno
war tief beeindruckt und wurde mehrfach zu poetischen Worten angeregt darüber,
wie für diese Stelle \enquote{die Sprache schlechterdings keinen anderen
Begriff darbietet als den des Erhabenen}\cite[S.~261]{adorno}; wie
sie den \enquote{Charakter des \enquote{Sterns}} trägt\cite[Nr. 357]{adorno};
oder wie ihre \enquote{Schönheit} und \enquote{geistige Kraft des
Zuspruchs} zu ergründen sei\cite[Fußnoten 290 und 291]{adorno}.

Formal ist es eine Analogie zur Episode \enquote{im alten Stil} des ersten
Satzes: im dritten Teil der Durchführung wird ein ganz neues Material
oder ganz neue Stilistik eingeführt, die für den Rest des Satzes quasi
ohne Funktion bleiben.  Eine wunderschöne Melodie, \textit{molto cantabile},
ist in der ersten Geige aufgespannt; die Begleitfiguren werden zu
32tel-Sextolen diminuiert und entfalten dadurch viel stärker atmosphärische
Wirkung; und im schlichten Bass des Cellos reicht Beethoven’s Legato-Bogen
über vier lange Takte.

In T.~76 beginnt ein Quintstieg uns langsam aus dieser traumhaften
Stimmung herauszuheben; sein letzter Schritt von C- nach \tonart{G-Dur}
erhält größte Emphase und strahlt durch \lilyDynamics{sfp}, Parallelbewegung
aller vier Instrumente und Quintlage.  Aber letztlich ist \tonart{C-Dur}
doch die Vorbereitung für das \tonart{f-moll} der Reprise, und nach
zwei Takten auskomponierten \textit{diminuendo}s hebt die erste Geige
mit einer verzierten Fassung des Hauptthemas wieder an – ein Zug von
\enquote{entwickelnder Variation} kommt zum Vorschein.

Auch die Begleitung ist ungleich komplexer als zuvor, mit vier
selbständigen rhythmischen Schichten und raffinierter Klanglichkeit
(\textit{pizzicato} im Cello).  Der Beginn der Überleitung bekommt
eine andere Gestalt, aber im weiteren verläuft die Reprise entlang der
vorgezeichneten Bahnen.\footnote{Wenn Beethoven zu T.~109 das
d\textsuperscript{4} vermeidet, ist das klangliche Absicht oder
Rücksichtnahme auf die Geiger? Für die Aufführungspraxis durchaus
eine interessante Frage.}

Die Coda präsentiert noch einmal das Hauptthema in Gänze, in einer
dritten Variation.  Lediglich der letzte \textit{morendo}-Takt wird
unter Verwendung der Seitensatzrhythmik anders weitergeführt, mit
konvergierenden Linien, deren Ziel in T.~124f. eine Variante der
überleitenden Takte~19f. ist.  Auf dem \tonart{G-Dur}-Terzquartakkord
friert die Bewegung plötzlich ein und die erste Geige schwingt sich
auf zu einer Vierundsechzigstel-Kadenz phantastischen Ausmaßes\footnote{Dies
nicht nur in der Horizontalen, sondern auch indem sie den ganzen
verfügbaren Tonraum nutzt.}; Kerman schreibt über den \foreignquote{english}
{sense of almost physical release}\cite[p.~112]{kerman}, der sich hier
einstellt.

\begin{notenbeispiel}
 \includegraphics{nbs/3-parallelen-zu-op-61.pdf}
 \caption{Illustration der motivischen Parallelen zwischen op.~59/1 und op.~61}
 \label{nbs:3-parall}
\end{notenbeispiel}

Erstaunlicherweise scheint ein Aspekt dieser Stelle in der Literatur
noch gar keine Beachtung gefunden zu haben: Nicht nur bricht hier
eine konzertante Virtuosität hervor, die sonst nicht für Beethoven’s
Quartettstil typisch ist, sondern Teile der Kadenz sind auch motivisch
sehr nah zu Stellen aus dem entstehungszeitlich benachbarten
Violinkonzert op.~61, wie in Notenbeispiel \vref{nbs:3-parall} zu
sehen. – Die steigenden Tonleitern in T.~129 werden durch fallende
Tonleitern der übrigen Instrumente kontrapunktiert und führen zum
extrem dissoziierten Zusammenklang von groß~C im Cello und c\textsuperscript{4}
in der ersten Geige in der Mitte von T.~130.  Dann pendeln sich die
Vierundsechzigstel auf c$''$ ein und nahtlos eröffnet das Cello den
vierten Satz.