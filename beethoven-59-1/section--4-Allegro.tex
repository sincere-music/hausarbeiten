\section{IV. Thème russe. Allegro}

Man wüsste gerne, welche Wege Beethoven’s Kreativität gegangen
ist, als er beschloss, dieses «\,Thème russe\,» als Thema für
den vierten Satz dieses Quartetts heranzuziehen.  Denn in
der Sammlung, aus der er es entnahm, ist es zitiert als
\enquote{Soldatenklage}, im Molto Andante und in \tonart{g-moll};
den Textanfang zitiert Nancy November als \foreignquote{english}{O
misfortune mine}.\cite[p.~83]{november}  Eine Quinte nach oben
transponiert, nach Dur umharmonisiert, stark beschleunigt und
humorvoll artikuliert ist es kaum wiederzuerkennen – ob das von
Beethoven beabsichtigt oder Ergebnis eines unbewussten kreativen
Prozesses ist, darüber lässt sich nur spekulieren.  In gleicher
Weise lässt sich fragen, ob die Entwicklung des ganzen Quartetts
aus dieser Keimzelle kontingent oder mit einem teleologischen
Plan geschah, oder ob wir in der geschilderten Transformation
im Gegenteil einen Hinweis sehen, dass Beethoven das Thema in
eine bestehende Konzeption einfügte.

Alle Sätze des Quartetts folgen mehr oder weniger eng einer
Sonatensatzform, der letzte ist darin nächst dem dritten noch
am \enquote*{konservativsten}, und als einziger besitzt er
auch eine wiederholte Exposition.  Wie im ersten Satz beobachten
wir auch hier, dass sich das Thema gewissermaßen schrittweise
manifestiert und hier ist es erst in T.~18, dass wir ein stabiles
\tonart{F-Dur} erreichen und das Thema in einer gewöhnlichen
Harmonisierung auftritt.  Dann beginnt es sich aber vier Takte
später auch schon wieder in Motive aufzulösen und nimmt stärker
überleitungsartigen Charakter an, bevor wir in T.~27 zunächst
in den Mittelstimmen ein eigenes Überleitungsmotiv sehen.  Mit
gesteigertem Elan stürzt sich die Musik ab T.~40 in die Modulation,
und der Seitensatz trägt mit seiner schlüssigen Periodik und
vibrierenden Begleitung echten Finalcharakter.

Hier ist nun der rechte Platz für eine Verunsicherung, die
Beethoven durch eine Mollvariante und kanonische Führung mit
\enquote{oktavigen Unreinheiten}\cite[Nr. 262]{adorno}
herbeiführt.  Diese Mollvariante mit ihrer verminderten Septime
hat bereits klagenden Charakter; mit den drei \lilyDynamics{pp}-Takten
62\,ff. wird die Stimmung nachgerade ängstlich.  Aber dem wird
jetzt kein großer Raum mehr gelassen und \lilyDynamics{ff} werden
diese \enquote{Zweifel} energisch zerstreut, sodass die Wiederkehr
des \lilyDynamics{pp} auf einem verminderten Septakkord schon
fast humoristischen Charakter hat.  Hiermit ist die dritte thematische
(Schluss-)Gruppe vorbereitet, die nach dem \textit{poco rit.} direkt
durchstartet; durch staccato und versetzte Punktierungen ist der
vorherige Elan noch gesteigert und führt zuletzt mit
\enquote{lombardischem Übermut} ins \lilyDynamics{ff}, wo Beethoven
enthüllt, dass die beiden letzten Sätze nicht nur durch Kadenz und
\textit{attacca}-Übergang verknüpft sind: T.~89–99 sind eine hinreißende
vierstimmige Erweiterung des Schlusses der Kadenz, die entsprechend
wirkungsvoll zurück in die Exposition respektive weiter in die
Durchführung leitet.

Mit einer tonal verfremdeten, verkürzten Fassung der ersten
Themengruppe hebt ganz typisch die Durchführung an; ebenso typisch
sind die Modulationen, Motivabspaltung und der Schwerpunkt von
T.~123–167 auf \tonart{d-moll}.  Die deutlichste Binnengliederung
ergibt sich mit T.~141, wo die dritte Themengruppe übernimmt in
einem Abschnitt, den man eindeutig bis einschließlich T.~176 fassen
würde, käme nicht zwischendurch (T.~157\,ff.) fast ritornellartig
wieder der \enquote{Kadenzschluss} in einer \tonart{d-moll}-Variante.  So
könnte man sagen, dass sich ab T.~165, oder ab T.~177 eine Art
Niemandsland erstreckt.  Denn gemeinsam mit dem ersten und zweiten
Satz hat der vierte die verschleierte Reprise, hier vor allem dadurch,
dass wir uns T.~179 noch in B-Dur befinden.  Wie sich also bereits
beim Satzbeginn das Thema allmählich manifestierte, so ist genau
analog erst T.~190 der Punkt, wo wir ganz in der Reprise angekommen
sind.

Ab hier verläuft auch bei aller Kunstfertigkeit im Stimmentausch
und dem notwendigerweise veränderten Modulationsweg zum zweiten
Thema die Reprise genau parallel zur Exposition, bis zu dem Punkt,
wo entsprechend wieder das \enquote{Ritornell} käme.  Hier folgt
eine trugschlüssige Ausweichung und zwei Akkorde werden \lilyDynamics{ff}
jeweils zu einer Fermate hin aufgetürmt, beim zweiten Mal noch
durch ein kadenzartiges fallendes Arpeggio der ersten Geige
abgerundet; ein großer formaler Einschnitt grenzt also die Coda
des letzten Satzes ab.

Der imitatorische Abschnitt, der nun folgt, ist gegenüber denen
der vorigen Sätze (I/185\,ff., II/404\,ff.) kleingliedriger; er hebt
\lilyDynamics{pp} an, bleibt dreimal zweistimmig und verdichtet sich
dann rasch in der Vertikalen und in der Einsatzfolge.  Nach einer
kurzlebigen Ausweichung Richtung \tonart{f-moll} beginnen
Sequenzmodelle, die wieder größere Bögen öffnen.  In T.~285 beginnt
eine freiere Abwandlung des \enquote{Ritornells}, diesmal nach oben
statt nach unten gerichtet sowie mit treibenden Punktierungen
grundiert.  Hier fallen alle vier Instrumente in den Triller ein
und ab da wird die Musik zunehmend bacchantisch, \textit{stretta}-artig
im Charakter, um plötzlich noch einmal auf einem \tonart{B-Dur}-Akkord
einzuhalten.

Es ist im Grunde auch ein Beethoven-typischer Kunstgriff, wie hier
kurz vor Schluss noch ein Adagio-Abschnitt eingefügt wird, der in
höchster Lage noch einmal das \textit{Thème russe} zitiert.  Zugleich
ist hier die nächste Annäherung an das klagende Moll der Originalgestalt
gefunden, welches aber die allgemein gelöste Stimmung nur unterbricht und
nicht gefährdet.  \textit{sempre perdendosi} versickert das harmonische
Pendel, bevor der Satz presto und \lilyDynamics{ff} ausgekehrt wird.