\section{Summa}

Was lässt sich nun mit dem Hintergrund all dieser Überlegungen über das
Quartett als Ganzes sagen?

Die Grundtonart \tonart{F-Dur} ist eine Natur- und Pastoraltonart,
was sich musikalisch besonders in den rhapsodischen Passagen
und Hörnerimitationen des ersten Satzes zeigt.  Der
freudige Grundcharakter des Stückes bleibt immer friedlich; bereits dem
ersten Satz mit der weiten Ausdehnung seiner Themen und noch mehr dem
letzten Satz ist eine große Leichtigkeit eigen, die sich auch in dem
Reichtum an Ideen und ihrer bisweilen recht losen, spielerischen Reihung
niederschlägt.  Auch das Adagio ist mit dem Bild des \enquote{Trauerweiden
oder Akazien-Baumes}, das ihm von seiner Konzeption her eingeschrieben ist, 
naturverbunden; diese Verbindung von Natur und subjektivem Gefühlsausdruck
erinnert durchaus auch an romantische Topoi, trotz aller Vorsicht, die
bei Beethoven mit solchen Begriffen geboten ist.  Vielleicht lässt sich
in Bezug auf das Werk in seiner Endgestalt von einer unterschwelligen
Tendenz dazu sprechen.

Eine solcherart friedliche Grundhaltung findet sich bei Beethoven im Jahr
1806 ziemlich häufig, neben op. 59/1 auch in den beiden Konzerten op.~58
und 61.  Der ausgedehnte Charakter der jeweiligen Hauptthemen wurde schon
erwähnt; wir können hier auch auf Adorno’s Begriff des \enquote{extensiven
Typs} verweisen, unter dem er diese drei Werke, neben anderen,
fasst\cite[Nr.~216 \& 219]{adorno}.

Die Modulationen des ganzen Werks zeigen eine bemerkenswerte Neigung
zur Seite der B-Tonarten.  Nicht nur nimmt \tonart{Des-Dur} mit seinem
sehnsüchtig-feierlichen Charakter wie bereits erwähnt einen breiten
Raum in den ersten drei Sätzen ein, auch wird die Fremdartigkeit der
Fugenepisode im ersten Satz durch die extreme Tonart \tonart{es-moll}
unterstrichen.  Kreuztonarten tauchen immer nur kurz auf, im ersten
Satz nur mit dem \tonart{a-moll} und \tonart{e-moll} T.~230f.  Bereits
im zweiten Satz ist die Kontrastwirkung des jeweils überraschenden
\tonart{H-Dur} und \tonart{e-moll} in T.~171 resp. 450 deutlich
stärker, und im dritten Satz bricht das strahlende \tonart{D-Dur}
hochdramatisch aus der düsteren Umgebung heraus.

Irritierend ist beim Blick auf die Großform und die Beziehungen der
Sätze untereinander zunächst der diametrale Gegensatz der beiden
Mittelsätze.  Wie hält Beethoven diese Diskrepanz zusammen?

Zum einen bilden die beiden Außensätze eine Klammer, durch die Gemeinsamkeit
in Tonart und somit Grundcharakter wie auch durch die verwandten Anfänge;
zum anderen lässt sich ein Schlüssel finden in der Vielfalt der Charaktere,
die sich im ersten Satz abwechseln.  Denn sowohl das Spielerische,
Irrlichternde des Allegretto scherzando als auch der Schmerz\footnote{Hierdurch
findet die Episode vor dem Seitensatz (I/53) ihre nachträgliche Erklärung.} und die
Empfindsamkeit des Adagio sind hier enthalten, und werden im folgenden quasi
entfaltet und ins Extrem gesteigert, bevor die Kadenz der ersten Geige den
Weg zurück in die grundlegende Fröhlichkeit des Quartetts bahnt.  Und so werden
die Gegensätze versöhnt.