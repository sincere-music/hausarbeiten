\documentclass[a4paper,11pt,twoside]{scrartcl}

\usepackage{verbatim}
\usepackage[quiet]{polyglossia}
\usepackage[autostyle,german=guillemets,french=guillemets,maxlevel=4]{csquotes}
\usepackage{fontspec}
\usepackage{lilyglyphs}
\usepackage{textcomp}
\usepackage[
	backend=biber,
	style=footnote-dw,
%% biblatex-dw-Optionen %%%
%	acronyms=true,
%	addyear=true,
	annotation=true,
%	citeauthor=normalfont,%namefontfoot,%namefont,
%	citedas=false,
%	citepages=suppress,%omit,%permit,%separate,
%	edbyidem=false,
%	editorstring=normal,%brackets,
%	edstringincitations=false,
%	edsuper=true,
%	firstfull=true,
%	firstfullname=true,
%	firstnamefont=smallcaps,%italic,%bold,
%	ibidemfont=smallcaps,%italic,%bold,
%	idembib=false,
%	idembibformat=dash,
%	idemfont=smallcaps,%italic,%bold,
%	inreference=full,%normal,
%	journalnumber=afteryear,%date,%standard
	library=true,
%	namefont=smallcaps,%italic,%bold,
%	nolocation=true,
%	nopublisher=false,
%	oldauthor=false,
%	omiteditor=true,
%	origfields=false,
%	origfieldsformat=parens,%brackets,%punct,
%	pagetotal=true,
%	pseudoauthor=false,
%	series=afteryear,
%	shorthandibid=false,
%	shorthandinbib=true,
%	shorthandwidth=40pt,%3em,
%	shortjournal=true,
%	terselos=false,
%	xref=true,
%% biblatex-Optionen %%%
%	autocite=plain,
%	citetracker=false,
%	doi=true,
%	eprint=true,
%	ibidpage=true,
%	ibidtracker=false,
%	idemtracker=false,
%	isbn=true,
%	pagetracker=false,
	hyperref
]{biblatex}
\usepackage[hidelinks]{hyperref}
\usepackage[verbose=silent]{microtype}
\usepackage{rotating}

%% bessere Querverweise %%%%%%%%%%%%%%%%
\usepackage{varioref}
% siehe Packetdoku S. 14
\def\reftextfaceafter {auf der \reftextvario
{gegenüberliegenden}{gegenüberliegenden} Seite}%
\def\reftextfacebefore {auf der \reftextvario
{gegenüberliegenden}{gegenüberliegenden} Seite}%
\def\reftextafter
{auf der \reftextvario
{nächsten}{folgenden} Seite}%
\def\reftextbefore
{auf der \reftextvario
{vorigen}{vorhergehenden} Seite}%
\def\reftextcurrent
{\reftextvario
{auf dieser}{diese} Seite}%

%% Abkürzungen
\newcommand{\Ar}{\textrightarrow}
\newcommand{\Nr}[1]{\textmd{\textit{Nr.~#1}}}

%% Formatierung subsubsection header %%%%
%%%%%%%%%%%%%%%%%%%%%%%%%%%%%%%%%%%%%%%%%
\setkomafont{subsubsection}{\rmfamily\raggedleft}
%%%%%%%%%%%%%%%%%%%%%%%%%%%%%%%%%%%%%%%%%

\bibliography{quellen}

\setdefaultlanguage{german}
\setotherlanguage[variant=ancient]{greek}
\newfontfamily\greekfont{CMUSerif-Roman}
\setotherlanguage[variant=british]{english}

\title{Prophet, Lehrer, Freund, Beter}
\subtitle{Die Darstellung Jesu\\in den Rezitativen der Matthäuspassion von Seb. Bach}
\author{Simon Albrecht}

\begin{document}
\maketitle

Johann Sebastian Bach’s Matthäuspassion und ihr Libretto von
Christian Friedrich Henrici befinden sich in einem Spannungsfeld
von orthodox-dogmatischer Predigt und affektbetonter, mitunter
theatralischer Darstellung.  Im Text finden sich allenthalben
Verweise auf Autoren wie David Hollatz, Johann Gerhard und Leonhard
Hütter, genauso aber Zitate wie \enquote{Ich will dir mein Herze
schenken} oder \enquote{Ich will Jesum selbst begraben}, an denen
die emotionale Beteiligung des gedachten Hörers ablesbar wird.  Dies
betrifft zum einen die madrigalische Dichtung Henrici’s, zum anderen
ist aber auch in der biblischen Überlieferung selbst die Person Jesu
Christi mehrschichtig: Manche seiner Worte sind in göttlicher oder
prophetischer Kapazität gesprochen, andere sind Worte eines Lehrers
oder Freundes zu seinen Jüngern, und dann hören wir zunehmend den
Menschen Jesus, der in äußerstem Leiden zu Gott betet.  Im Folgenden
soll genauer untersucht werden, wie sich diese Facetten in der
Portraitierung Jesu Christi in den Rezitativen niederschlagen und
welches Licht die musikalische Darstellung darauf wirft.

\section{Die Rollen Jesu im Passionsbericht nach Matthäus}

\subsection{Überblick}

Die biblische Textgrundlage für die Matthäuspassion besteht in den
Kapiteln 26 und 27 des Evangeliums.  Darin eingeschlossen sind insgesamt
22 Worte Jesu\footnote{Von diesen zählt nur das letzte (\enquote{Eli,
Eli}) zu den \enquote{Sieben Worten Jesu am Kreuz}, deren restliche sechs
sind den anderen Evangelien entnommen.}.  An der zeitlichen
Dichte dieser Worte lässt sich eine deutliche Zweiteilung ablesen,
die mit den zwei Teilen der Vertonung zusammenfällt: Mit den Worten
\enquote{Aber das ist alles geschehen, dass erfüllt würden die Schriften
der Propheten} (Mt 26,56 – Nr. 28, T.25\,ff.) gibt Jesus das Heft
des Handelns aus der Hand, unterwirft sich dem Leiden, und ergreift
danach bis zum Tod am Kreuz nur noch dreimal das Wort, während die
19~Christusworte im ersten Teil recht dicht gesät sind.  Tabelle
\vref{tab:overview} gibt einen Überblick.

\begin{sidewaystable}
 \centering
 \caption{Übersicht über die Jesusworte in der Matthäuspassion}
 \begin{tabular}{r@{}l|r@{}l|l|l|l}
  \multicolumn{2}{l}{Nr.\footnote{Nummerierung nach NBA.}} &\multicolumn{2}{l}{Vers (Mt)} &Tonarten &Textanfang &Rolle(n) \\
  \hline
  2  &   &26, &2       &G\Ar{}h       &\enquote{Ihr wisset, dass nach zweien…} &Prophet (aber affektives „gekreuziget“!) \\
  4  &e. &    &10b–13  &F–g–d–a–e     &\enquote{Was bekümmert ihr das Weib?}   &Lehrer, Prophet \\
  9  &c. &    &18      &G\Ar{}e       &\enquote{Gehet hin in die Stadt}        &Lehrer \\
     &   &    &21b     &Es\Ar{}c      &\enquote{Wahrlich, ich sage euch: Einer unter euch} &Prophet \\
  11 &   &    &23b–24  &f\Ar{}Es      &\enquote{Der mit der Hand}              &Prophet (einschließlich des \enquote{Wehe!}) \\
     &   &    &25b     &g             &\enquote{Du sagests.}                   &Prophet \\
     &   &    &26b     &F             &\enquote{Nehmet, esset}                 &Prophet?, Lehrer \\
     &   &    &27b–29  &C\Ar{}G       &\enquote{Trinket alle daraus}           &Prophet, (Lehrer) \\
  14 &   &    &31b.32  &(A)fis\Ar{}E  &\enquote{In dieser Nacht werdet ihr}    &Prophet, (Lehrer) \\
  16 &   &    &34b     &D\Ar{}e       &\enquote{Wahrlich, ich sage dir: In dieser Nacht} &Prophet \\
  18 &   &    &36b     &B\Ar{}Es      &\enquote{Setzet euch hie}               &Freund \\
     &   &    &38b     &c\Ar{}As      &\enquote{Meine Seele ist betrübt}       &Freund \\
  21 &   &    &39b     &Es\Ar{}g      &\enquote{Mein Vater, ists möglich}      &Beter \\
  24 &   &    &40b.41  &F\Ar{}g\Ar{}a &\enquote{Könnet ihr denn nicht…}        &Freund, Lehrer \\
     &   &    &42b     &e\Ar{}h       &\enquote{Mein Vater, ists nicht möglich}&Beter \\
  26 &   &    &45b.46  &fis\Ar{}gis   &\enquote{Ach, wollt ihr nun schlafen}   &Freund, Prophet \\
     &   &    &50a     &D             &\enquote{Mein Freund, warum bist du kommen?} &(\enquote*{Über}-)Freund \\
  28 &   &    &52–54   &A\Ar{}fis     &\enquote{Stecke dein Schwert an seinen Ort} &Lehrer, Prophet \\
     &   &    &55b–56a &h–cis–E–A     &\enquote{Ihr seid ausgegangen}          &Lehrer, (Prophet) \\
  \hline
  36 &a. &    &64      &h\Ar{}e       &\enquote{Du sagests. … Von nun an wirds geschehen} &Prophet \\
  43 &   &27, &11b     &c             &\enquote{Du sagests.}                   &Prophet \\
  61 &a. &    &46b     &b             &\enquote{Eli, Eli}                      &Beter
 \end{tabular}
 \label{tab:overview}
\end{sidewaystable}

Wenn sich die Rollen natürlich nicht immer voneinander trennen lassen,
so sieht man doch, dass die Rolle des Propheten deutlich den meisten Raum
einnimmt, mit 14 Nennungen, gefolgt von der des Lehrers, mit 7
Nennungen.  Ganz anders die Geth\-se\-ma\-ne-Sze\-ne, in der Jesus vor
allem als Freund und als Beter auftritt.

In der Rolle des Propheten äußert sich am meisten nur göttliche Natur;
gewissermaßen von einer höheren Warte gesprochen, sind diese Worte kaum
affektiv geprägt und der menschlichen Logik enthoben.  Als Lehrer oder
Rabbi sind Jesu Worte bereits mehr in menschlichen Beziehungen verwurzelt,
zugleich aber durch formale Autorität gestützt.  Diese Formalität spielt
in der Geth\-se\-ma\-ne-Sze\-ne keine Rolle, wo Jesus seinen Freunden
die Bedrängnis seines Herzens offenbart und um ihren Beistand bittet.  Zuletzt
in der Rolle des Beters ist Jesus am Tiefpunkt des menschlichen Daseins
angekommen, wo er mit seinem Schicksal hadert und am Ende seine Gottverlassenheit
beklagt.

\subsection{Einzelbetrachtungen}
Das erste Jesuswort ist eine prophetische Ankündigung; es geht zwar um
sein eigenes künftiges Leiden, aber Jesus spricht in der dritten Person
und ohne persönliche Betroffenheit.

Das zweite Wort, \enquote{Was bekümmert ihr das Weib?}, ist ein gutes
Beispiel, wie die Rollen ineinander übergehen: Die ersten beiden Verse
sind ganz im Stil der früheren Lehrtätigkeit Jesu gehalten, dann aber
weitet er den Blick mit den zunehmend prophetischen Worten, die auf
die kommende Leidensgeschichte sowie die künftige Wirkung des
Evangeliums verweisen.

Bei dem Wort: \enquote{Der mit der Hand mit mir in die Schüssel tauchet,
der wird mich verraten} \&c. ist aufschlussreich zu beobachten, dass es
sich um ein prophetisches Wort handelt, denn damit ist das \enquote{wehe}
an den Verräter keine persönliche Drohung, sondern eine Verkündung
göttlichen Ratschlusses, welche Empathie von Jesus zulässt.

Bei der Einsetzung des Abendmahles durchdringen sich eine alltägliche
Situation, in der der Rabbi das Dankgebet spricht, seinen Jüngern das
Brot bricht und den Kelch reicht, sowie die Einsetzung des Abendmahles
als Ritus, in dem Leib und Blut Christi gegenwärtig sind\footnote{Kann
man hier womöglich von einer priesterlichen Rolle Jesu sprechen?}, und
zuletzt die Prophetie der neuen Gemeinschaft im Reich Gottes.

Eine merkliche Dramatisierung zeigt bereits der Text, als Jesus in
Gethsemane zum dritten Mal vom Gebet zurück zu seinen Jüngern kommt:
zunächst setzt er mit gesteigertem Affekt (\enquote{Ach}, V.~45) die
freundschaftliche Anrede von zuvor fort, dann folgt unvermittelt der
Sprung ins Prophetische: \enquote{Siehe, die Stunde ist da}.

Das direkt folgende Wort \enquote{Mein Freund, warum bist du kommen?}
ist ein weiterer Sonderfall. Oberflächlich ist es von Jesus als Freund
gesprochen und lädt den Verräter mit offenen Armen zur Umkehr ein,
besonders in Luther’s originaler Textfassung, die Bach vertont.  Der
griechische Urtext\footnote{zitiert nach \cite{nestle-aland}}
\foreignquote{greek}{ἑταῖ\-ρε, ἐφ’ ὃ πάρ\-ει} ist aber sowohl mit Punkt als
auch mit Fragezeichen überliefert, und viele Übersetzungen klingen
deutlich konfrontativer: \enquote{Amice, ad quod venisti!} (Vulgata),
\enquote{Freund, dazu bist du gekommen!} (Zürcher Bibel\cite{zuercher}),
und auch die 1984’er Revision der Luther-Bibel: \enquote{Mein Freund,
dazu bist du gekommen?}.  Von einem menschlichen Standpunkt her ist
die so interpretierte Reaktion wesentlich leichter nachzuvollziehen:
\enquote{Du bist doch nicht deswegen hergekommen!}, \enquote{Ich weiß
doch, dass du mich verraten hast!}.

Es folgen im zweiten Teil zwei Worte, die Jesus in der Vernehmung vor
Pilatus sagt, und in denen er bestätigt, der Christus und der König
der Juden zu sein.  Aus der sonstigen Wortkargheit Jesu in dieser
Szene sticht besonders hervor die Prophezeiung \enquote{Von nun an
werdet ihr sehen des Menschen Sohn sitzen zur Rechten der Kraft und
kommen in den Wolken des Himmels}, die auch inhaltlich aus dem
Gesprächskontext herausgelöst erscheint.  In dieser Situation agiert
Jesus ganz aus menschlichen Zwängen herausgehoben, wie entrückt.

Zuletzt das \enquote{Eli, Eli} ist das genaue Gegenteil.  Zwar nimmt
es als hebräisches Zitat aus Psalm~22 eine herausragende Stellung im
Text ein, aber hier spricht Jesus ganz als Mensch, den irdischen
Bedingungen völlig unterworfen.

\section{Die musikalische Umsetzung der Jesusworte}

\subsection{Kompositorische Grundstruktur}
Wie in allen oratorischen Werken Bach’s sind auch hier die Jesusworte
gegenüber dem Rest des Evangelientextes durch die Begleitung mit
Streichern ausgezeichnet.  Auf einer untersten Ebene handelt es sich
hierbei um die akkordische Aussetzung des Generalbasses, die aber
durchgehend als kunstvoller fünfstimmiger (unter Einbeziehung der
Singstimme) Satz gearbeitet ist, bis dahin, dass immer wieder
wichtige Akkordtöne in der Singstimme liegend von keinem der (obligaten)
Instrumente verdoppelt werden.  Außerdem wird die Bewegung der Streicher
oft (über das harmonische Tempo hinaus) diminuiert und zur affektmäßigen
Ausdeutung genutzt; dies vor allem zu Ende der Worte/Sätze und auffällig
oft auch mit dem Versschluss in eins fallend (z.B. Nr.~4e., T.~43\,f. oder
Nr.~28, T.~9).  In gewisser Weise ist dies ein Zwischentyp zwischen
\enquote*{secco}-~und \enquote*{accompagnato}-Rezitativ.

Die Singstimme selbst ist, wie Konrad Küster\cite{kuester} beschreibt,
in Anlehnung an ein Rezitationsmodell gestaltet, das sich ebenfalls
an der Versstruktur orientiert.  Dabei werden zwei verschiedene
Eingangsfloskeln – ein Sextakkord abwärts und ein Quartsextakkord
aufwärts\footnote{Exemplarisch dafür jeweils der erste Einsatz des
Evangelisten beziehungsweise Jesu.} – in vielfältiger Weise an die
Anforderungen der Deklamation sowie an geforderte/erwünschte Lagen
angepasst.

Die Einsetzungsworte des Abendmahls in Nr.~11 gestaltet Bach im Grunde
als eigene geschlossene Nummern, als Ariosi.  In verschiedenen
Abstufungen ist diese Tendenz auch bei anderen Worten zu beobachten:
zum Beispiel bei \enquote{Eli, Eli}, mit \textit{adagio} markiert, oder
\enquote{Meine Seele ist betrübt} (Nr.~18, T.~11\,ff.), wo das Bogenvibrato
des folgenden Recitativo accompagnato bereits vorweggenommen wird.  Auch
das \textit{vivace} und \textit{moderato} in Nr.~14 ab T.~8 fällt aus
dem üblichen Gestus heraus, indem sich das alttestamentarische Zitat
als \enquote{echtes} Accompagnato herausbildet.

\subsection{Einzelbetrachtungen}
\subsubsection*{\Nr{2} \enquote{Ihr wisset, daß nach zweien Tagen Ostern wird}}
Nach den harmonischen Turbulenzen und dem sehr ernsthaften e-moll des
monumentalen Eingangschores setzt Bach den Beginn des Evangelienberichts
in G-Dur deutlich dagegen ab.  Auch ist das typische harmonische
Modell über einem absteigenden Bass (exemplarisch verwirklicht
in Nr.~9a) hier zunächst maximal ausgedehnt, indem erst nach fast drei
Takten des Jesuswortes die zweite Bassnote folgt.  Die anstehende
Festlichkeit des Passa wird durch leuchtendes A-Dur gezeichnet, was
auch die Deutlichkeit des dramatischen Reliefs erhöht: den
Wendepunkt markiert Bach mit der hyperbolischen Verzierung auf
\enquote{\underline{ü}berantwortet werden}, bevor \enquote{daß er
gekreuziget werde} mit schmerzlicher Melodik und Harmonik untermalt
wird und nach h-moll moduliert.  Im Einzelnen sind es der passus
duriusculus, viele Vorhaltbildungen und der relativ eindeutige Chiasmus
in der Sechzehntelfigur a-fis-g-e, die als Figuren eingesetzt sind.

Bach fügt also dem prophetischen Wort eine expressive Ebene hinzu,
die im Text nicht angelegt ist; die Klage ist mehr die Klage des
Betrachters als die Jesu selbst.

\subsubsection*{\Nr{4e.} \enquote{Was bekümmert ihr das Weib?}}
Auch im folgenden Wort sind Affekte eingeführt, die als Jesu eigene
zu betrachten im Bibeltext kein Anlass besteht.  Wo Jesus als Lehrer
und Prophet spricht, gibt Bach’s Musikalisierung eine große Bildhaftigkeit
und Empathie hinzu, so mit der \enquote{wehklagenden}\cite[2.~Theil,
1.~Abtheilung, S.~104]{kirnberger} verminderten Septime auf
\enquote{\underline{Ar}men} und mit den fallenden Seufzern beim
Verweis auf Jesu Begräbnis.

Die Tonartenwahl passt hier hervorragend zu den Beschreibungen
von Johann Mattheson\cite[S.~236–252]{mattheson}, der F-Dur als
\enquote{capable, die schönsten Sentiments von der Welt zu
exprimieren, es sey nun Großmuth\,/\,Standthafftigkeit\,/\,Liebe…}
beschreibt.  Jesus ist also vom missgünstigen Eifer der Jünger ganz
unbeeindruckt – oder vielmehr: dem barocken Zuhörer wird die
Großherzigkeit von Jesu Handeln vor Augen geführt –, und der
liebevolle Charakter bleibt auch im folgenden g-moll erhalten,
nun eingetrübt: laut Mattheson hat dieses eine \enquote{ziemliche
Ernsthafftigkeit mit einer muntern Lieblichkeit vermischet} und
ist zu \enquote{zärtliche[m]} sowie \enquote{beydes zu mäßigen
Klagen und temperirter Frölichkeit bequem}.

In den weiteren Takten lässt sich diese genaue Übereinstimmung
mit dem Tonartencharakter nicht beobachten, zum einen weil d-, a-,
e-moll darin weniger klar sind, zum anderen ist für Bach sicher
die Überleitung zum h-/fis-moll des folgenden Rezitativs/der Arie
Anlass genug.

Die Universalität der künftigen Verbreitung des Evangeliums
wird durch den Spitzenton e hervorgehoben.

\subsubsection*{\Nr{9c.} \enquote{Gehet hin in die Stadt zu einem}}
Die Rede Jesu beginnt sowohl melodisch wie harmonisch mit
den typischen Modellen, und auch in der verschachtelten wörtlichen
Rede (\enquote{sprecht zu ihm: \enquote{Der Meister lässt dir sagen:
\enquote{Meine Zeit ist hier}}}) wird jeder Beginn durch die
Dreiklangsmodelle markiert.  Durch das G-~und D-Dur ist die Stelle
heiter geprägt.

\subsubsection*{
\enquote{Wahrlich, ich sage euch: Einer unter euch wird mich verraten}
}
Der Evangelist moduliert ziemlich schnell nach Es-Dur, um die
folgende Prophezeiung vorzubereiten, welche sich dann nach c-moll
wendet.  Dieses beschreibt Mattheson als \enquote{überaus lieblich
dabey auch trist}, und in der Tat sehen wir auch in den weichen
Intervallen und Melodiebildungen, mit den Seufzern der ersten Geige,
dass Bach die Worte keineswegs als drohend interpretiert. Zugleich
finden wir die kleinen Septimen der Singstimme bei Kirnberger\cite[S.~104]{kirnberger}
als \enquote{etwas fürchterlich} charakterisiert – auch das passt
hier sehr gut.

\subsubsection*{\Nr{11} \enquote{Der mit der Hand mit mir in die Schüssel tauchet}}
Dieses Wort ist nicht, wie einige zuvor, in den Instrumenten durch
rhetorische Figuren geschmückt; sein Ausdruck rührt von einigen
\enquote{falschen} (d.i. verminderten und übermäßigen) Intervallen
in der Singstimme sowie vor allem von der herrschenden Tonart f-moll
her.  Mattheson beschreibt sie mit vielen und starken Worten:
\enquote{F.~moll. scheinet eine gelinde und gelassene / wiewol dabey
tieffe und schwere / mit etwas Verzweiflung vergesellschaffte /
tödliche Hertzens-Angst vorzustellen / und ist über die massen
beweglich.  Er drücket eine schwartze / hülflose Melancholie\footnote{War
sich Mattheson wohl der Tautologie in \enquote{schwartze Melancholie}
bewusst?} schön aus / und will dem Zuhörer bisweilen ein Grauen oder
einen Schauder verursachen.}

Wir haben oben bereits erwähnt, dass dieses \enquote{Wehe} wiederum
keine persönliche Drohung ist, und finden dies durch die weniger
harte als empfindsame Charakterisierung der Tonart bestätigt.  Die
\enquote{tödliche Hertzens-Angst} ist an dieser Stelle erneut weniger
Jesu eigene Äußerung als die Reaktion des barocken Interpreten.  Wir
sehen also, dass es in der Darstellung Jesu nicht primär um eine
psy\-cho\-lo\-gisch-dra\-ma\-tisch stimmige, \enquote{realistische}
Darstellung, sondern um die Anregung des Zuhörers zu einer emotionalen
Reaktion geht, wie dies auch Rolf Dammann beschreibt\cite[S.~224\,ff.]{dammann}.

\subsubsection*{\enquote{Du sagests.}}
Lapidar und knapp ist die folgende Antwort Jesu, nun aber wieder von
den Streichern rhetorisch untermalt durch eine fallende Seufzerkette
in g-moll über Orgelpunkt.  Wiederum ist der Affekt also klagend,
bereits etwas gemildert in der Tonartenwahl.

\subsubsection*{
\enquote{Nehmet, esset, das ist mein Leib.} \\
\enquote{Trinket alle daraus}
}
Vom C-Dur schreibt Mattheson, dass es trotz seiner \enquote{rude[n]
und freche[n] Eigenschaft} von einem \enquote{habile[n] Componist […]
auch in tendren Fällen} angebracht werden kann.  Im gegenwärtigen Fall
wird uns die Musik zunächst in F-Dur bekannt gemacht und dann nach C
versetzt; die Streicherbegleitung, der anmutige 6/4-Takt und die sehr
lineare, fließende Bewegung tragen ebenfalls dazu bei, dass keineswegs
ein \enquote{frecher} Eindruck entsteht; im Gegenteil ist die Heiterkeit
dieser Stelle, die sich in der nachfolgenden Arie fortsetzt, fast ohne
Parallele im gesamten Werk.\footnote{Auch die Arien \enquote{Sehet,
Jesus hat die Hand} und \enquote{Mache dich, mein Herze rein} sind
zumindest viel gedeckter gehalten in Tonarten und Instrumentation.}

Bemerkenswert am zweiten, längeren Arioso ist die sehr kontinuierliche
Bewegung; mit Ausnahme von T.~26 wird das Stück von nahezu durchgehenden
Achteln weitergetragen, und auch die große textliche Zäsur zu V.~29
in T.~31 wird durch die Begleitstimmen überbrückt.  Auch die vielen
Überbindungen, manche Synkopen und hemiolische Bildungen tragen zu
diesem großen Bogen bei, und vielleicht sehen wir darin eine
Veranschaulichung des zukunftsweisenden Charakters dieser Handlung:
hier wird etwas begonnen, das über die Kreuzigung hinaus durch alle
Zeiten Bestand hat.

\subsubsection*{\Nr{14} \enquote{In dieser Nacht}}
Nur an wenigen Stellen bietet der Passionsbericht Matthäi Anlass zu
echt hypotypotischen Figuren.  Neben den \enquote{Wolken des Himmels}
(Mt.~26,\,64, Nr.~36a.), dem Zerreißen des Vorhangs im Tempel und dem
Erdbeben treten sie vor allem hier in Nr.~14 gehäuft auf: der Weg
hinauf zum Ölberg wird durch die Tonleiter des Violoncellos in T.~2
gemalt; die zerstreuten Schafe im \textit{vivace} T.~8\,f.; und die
Auferstehung – \textit{moderato} – in den Geigen im Auftakt
zu 12 wie auch in der steigenden Gesangslinie.  Vielleicht ist es
dabei nicht übertrieben, die stac\-ca\-to-Ach\-tel am Ende als
Schritte zu verstehen.

Hingegen die Seufzer T.~6 und die verminderten Quinten der Singstimme
zuvor sind Ausdruck derselben \enquote{großen Betrübniß}\cite[S.~251]{mattheson},
die bereits in der Tonart fis-moll steckt.

\subsubsection*{\Nr{16} \enquote{Wahrlich, ich sage dir: In dieser Nacht}}
Diese Prophezeiung bewegt sich mit dem hauptsächlichen e-moll bereits
weiter in den Kreuzen als die vorherigen; doch auch diesem wird von
Mattheson keine Härte, sondern ein \enquote{tieffdenckend[er],
betrübt[er]} Affekt beigelegt.  Mit den Tonwiederholungen auf
\enquote{\underline{ehe der Hahn}} und den Sprüngen steckt zwar in
der Deklamation großer Nachdruck, aber auch hier \emph{verkündet}
Jesus eine Wahrheit, anstatt sie Petro etwa vorzuhalten.  Und
wiederum sind \enquote{falsche} Intervalle mehr dem Publikum zur
Verdeutlichung als zur psychologischen Interpretation der Handlung
bestimmt.

\subsubsection*{\Nr{18} \enquote{Setzet euch hie}}
Im zweiten Teil von Händel’s Messias können wir einen markanten
Gebrauch tonartlicher Extreme an dramaturgischen Schlüsselstellen
beobachten: dem b-moll zu Beginn von Nr.~24 (\foreignquote{english}
{All they, that see him, laugh him to scorn}) wird nach kurzer Zeit
das A-Dur von Nr.~29 \foreignquote{english}{But Thou didst not leave
His soul in hell} gegenübergestellt.  Ähnlich arbeitet Bach in der
Gethsemane-Szene der Matthäuspassion: Nr.~18, T.~9\,ff. sowie das
folgende Recitativo con Coro Nr.~19 sind in f-moll gehalten,
dann steigt die Tonalität dramatisch bis zum gis-moll (mit
doppeldominantischem cisis!\footnote{Wir können nur mutmaßen, wie
ein Sänger zu Bach’s Zeit auf eine solche Ungeheuerlichkeit reagiert
haben mag…}) in Nr.~26, T.~14–16 (\enquote{Siehe, er ist da, der
mich verrät!}).

Doch wir greifen voraus; zu Beginn der Szene wird noch eine große Ruhe
verströmt.  Sie äußert sich in der tiefen Lage, den Haltetönen und dem
langsamen Sprachrhythmus, der auf dem Wort \enquote{\underline{be}te}
ganz aufgehoben wird.  Wiederum ist es am Versende, dass Bach in den
Streicherstimmen selbständige, expressive Figuren einführt.

Nach dieser Stelle wird zum ersten Mal deutlich eine Gefühlsregung
Jesu geschildert, zunächst durch den Evangelisten mit den Worten
\enquote{und fing an zu trauern und zu zagen}, von Bach durch
Auflösung der syllabischen Diktion, Vorhalte, das neapolitanische
ges und verminderte Intervalle veranschaulicht.

\subsubsection*{\enquote{Meine Seele ist betrübt}}
Wie bereits erwähnt, ist dieses nächste Wort als Arioso komponiert.  Die
\enquote{Bebung} in den Streichern ist bereits starkes Ausdrucksmittel
für die innerliche Regung Jesu.  In der Singstimme werden die Wörter
\enquote{be\underline{trübt}}\footnote{Der verminderte Akkord in tiefer,
enger Lage taugt durchaus auch als Hypotypose für \enquote{trüb}.} und
\enquote{Tod} durch lange Haltetöne hervorgehoben; letzteres auch durch
die bildhaft sehr tiefe Lage.  Die Aufforderung \enquote{bleibet hie
und wachet mit mir!} wird stark kontrastiert durch plötzlich hohe Lage
und wieder rezitativischen Gestus; man könnte daher meinen, die ariosen
Worte zuvor seien mehr nach innen gerichtet.  Doch das wäre zu sehr
veristisch gedacht: die klar unterschiedene Darstellung hebt für den
Zuhörer das Relief hervor, ohne dass der innere Zusammenhang der
Satzhälften dadurch berührt wäre.

\subsubsection*{\Nr{21} \enquote{Mein Vater, ists möglich}}
In der Einleitung des Evangelisten sehen wir wie in Nr.~18 das Wort
\enquote{beten} aus dem rezitativischen Fluss herausgenommen und durch
eine ariose Figur in der Begleitung ausgestaltet.  Jesu folgende Worte
erhalten großen Nachdruck durch die hohe Lage und die prominente
synkopische Viertel auf \enquote{\underline{son}dern wie du willt}.  Die
Zweiteilung in die Bitte um Verschonung und die Hingabe in Gottes Willen
ist musikalisch umgesetzt, indem sich die erste Hälfte melodisch und
harmonisch (nach B-Dur) öffnet, während die zweite den melodischen
Bogen abwärts zu Ende führt und in g-moll schließt.

\subsubsection*{\Nr{24} \enquote{Könnet ihr denn nicht eine Stunde mit mir wachen?}}
Eine erneute eindringliche Aufforderung ist wiederum mit dem Spitzenton
es$'$ dargestellt, während bei \enquote{in Anfechtung fallet} die Linie
hypotypotisch fällt.  Die sentenzhafte Moral \enquote{Der Geist
ist willig, aber das Fleisch ist schwach} erhält rhetorisches Gewicht
durch die lange Pause vorher und auch die Pointe wird durch eine Pause
verdeutlicht, wie auch durch den Trugschluss aus aufeinanderfolgenden
Septakkorden, welche die Schwäche eindrücklich umsetzen, verstärkt durch
den Seufzer in der Kadenz.

\subsubsection*{\enquote{Mein Vater, ists nicht möglich}}
Für diese wiederholte Bitte übernimmt Bach die melodische Form aus
Nr.~21, aber enorm gesteigert durch die Versetzung von Es-Dur/B-Dur
nach e-moll/h-moll.  Außerdem wird der Bruch in der Mitte vergrößert,
indem nun das \enquote{ich trinke ihn denn} aus tiefer Lage neu anhebt
und so die Bitterkeit des Kelches spürbar macht.

\subsubsection*{\Nr{26} \enquote{Ach, wollt ihr nun schlafen und ruhen? Siehe…}}
Auch bei diesem Wort ist die im Text angelegte Steigerung zum vorherigen
musikalisch wirkungsvoll übersetzt: der Ausruf \enquote{Ach!} setzt
unvermittelt und sehr dissonant ein, der rhetorische Gestus ist von
vielen Sechzehnteln und Pausen durchsetzt.  Gemeinsam mit der Harmonik,
die immer noch eine Quinte ansteigt, und der beängstigenden Figur der
ersten Geige in T.~9 wird so die Zuspitzung der Dramatik deutlich.

\subsubsection*{\enquote{Mein Freund, warum bist du kommen?}}
In denkbar starkem Kontrast dazu steht diese Anrede Jesu an Judas. Die
Streicherbegleitung besteht lediglich aus einem gehaltenen Akkord,
kein Harmoniewechsel findet statt, und das D-Dur hat nach dem vorigen
Geschehen ebenfalls eine eindeutig friedfertige Wirkung.  Die Anrede
\enquote{Mein Freund} ist melismatisch geschmückt und der obligate
Quartvorhalt bekommt sehr viel Zeit.  Bach verstärkt also die oben
beschriebene Deutung Luther’s, nach der in diesem Wort keinerlei
Vorwurf oder Kritik steckt, und der dramaturgische Effekt ist ohne
Zweifel bestechend, indem am Höhepunkt der Dramatik kurz die Zeit
still steht.

\subsubsection*{\Nr{28} \enquote{Stecke dein Schwert an seinen Ort}}
Nach dem großen Lamento und Aufruhr der Nr.~27 will die Handlung in
hohem Tempo weitergehen: mit vielen Sechzehnteln schildert der
Evangelist, wie Petrus dem Malchus\footnote{Beide sind im Matthäusevangelium
nicht mit Namen genannt.} ein Ohr abhaut.  Mit den Worten \enquote{Da
sprach Jesus zu ihm:} verlangsamt Bach das Tempo auf die Hälfte\footnote{Beweis,
wie wichtig es ist, im Rezitativ die Notenwerte genau zu lesen.}
und stellt so drastisch das Innehalten dar.  Die folgende \enquote*{Strafandrohung}
für die Gewalt erhält großen Nachdruck durch melodische Steigerung und
nach wie vor gedämpftes Tempo; hier verschiebt Bach sogar den textlichen
Schwerpunkt gegen den Takt auf die 2 von T.~8, um den Worten das rechte
Gewicht zu verleihen.  Erneut wird zwischen die Verse ein kleines
\enquote{Zwischenspiel} der Streicher gefügt.

Zwischen der Möglichkeit, von \enquote{mehr denn zwölf Legion Engel}
gerettet zu werden, und dem vorgezeichneten Weg ans Kreuz ergibt sich
ein wirksamer Kontrast, den Bach umsetzt, indem sich Vers~53 in hoher
Lage der Singstimme nach einem strahlenden A-Dur wendet, und dann in
der großen Zäsur nach fis-moll und in tiefere Lage fällt.  Besonders
frappierend ist nach dem fragenden Halbschluss, dass \enquote{es muß
also gehen.} nicht etwa trotzig, oder entschlussfreudig, sondern mit
der Bitterkeit eines verminderten Septakkords ausgesprochen wird.

\subsubsection*{\enquote{Ihr seid ausgegangen}}
Dies war zu den Jüngern gesprochen; direkt darauf wendet sich Jesus
an die Schar, die \enquote{ausgegangen} ist, ihn \enquote{zu fahen}.  Die
großen Sprünge und hohen Spitzentöne könnten emotionaler Ausdruck eines
Jesu sein, der in großer Aufregung spricht; eher aber entspricht es
der barocken Konzeption, dass hiermit wiederum primär dem Publikum die
Unerhörtheit der Ereignisse verdeutlicht wird.  Bei dieser Rede ist
der Kontrast fast umgekehrt gestaltet als bei der vorherigen an die
Jünger: der erste Teil ist in Moll gehalten, der zweite wendet sich
bei viel weicherer Melodik nach A-Dur und zeigt ein ganz anderes
Sich-fügen in das vorgeschriebene Schicksal.  Auch hier fallen wiederum
die Pausen auf in T.~26\,f., die eine Beruhigung darstellen.  Vielleicht
desto weniger können die Jünger mit dieser \enquote{zufriedenen} Art
der Schicksalsergebenheit umgehen und ergreifen sämtlich die Flucht.

\subsubsection*{\Nr{36a.} \enquote{Du sagests. … Von nun an wirds geschehen}}
Der zweite Teil der Passion sieht Jesus nicht mehr als Akteur der
Handlung; was Henrici mit dem Oxymoron \enquote{verdienstlich Leiden}
(Nr.~20, T.~48\,ff.) benennt, besteht nun darin, dass Jesus den falschen
Vorwürfen nichts entgegensetzt.

Jesu Schweigen ist musikalisch an zwei Stellen besonders prägnant umgesetzt:
in Nr.~33, T.~16\,f. folgt auf die Frage des Pontifex zunächst eine sehr
beredte Viertelpause (die Interpreten tunlichst nicht übergehen sollten)
und beim Wort \enquote{schwieg \underline{stille}} endet der Evangelist
in sehr tiefer Lage und damit entsprechend leise.  Noch deutlicher ist
dies in Nr.~43, T.~24 bei \enquote{antwortete er \underline{nichts}}.

Als Jesus in der Anhörung sein Schweigen zum ersten Mal bricht, folgt
auf die Bestätigung, dass er Christus und Gottes Sohn sei, die
Prophezeiung über seine Himmelfahrt.  Musikalisch leitet Bach Jesu
Antwort mit einem besonderen Einfall ein: eine (kurze) Quintfallsequenz
von leitereigenen Septakkorden verbindet die Worte des Evangelisten
mit denen Jesu.

Wir hatten oben beschrieben, dass das prophetische Wort hier im Grunde
aus der Logik der Handlung herausgelöst ist; Bach aber stellt es durch
den lange gehaltenen Akkord der Streicher mehr als folgerichtige
Erläuterung des Gesagten dar.  Weithin bewegt sich dieses Wort in
e-moll, was mit Mattheson’s Charakterisierung dieses Tones schwer in
Einklang zu bringen ist; weniger als Betrübnis und Nachdenken scheint
es hier eine gewisse entrückte Majestät darzustellen.

Dass die Sechzehntel der Geigen, später auch von der Singstimme
übernommen, mit ihren Bögen bildhaft die Wolken zeichnen (mehr
Augenmusik als beim Hören deutlich), haben wir oben bereits
erwähnt.  Vielleicht wollte Bach auch durch das hohe dis und
den großen Sprung bei \enquote{\underline{se}hen} andeuten, dass
es kein gewöhnliches Sehen, kein irdischer Anblick ist, von dem
Jesus hier spricht.

\subsubsection*{\Nr{43} \enquote{Du sagests.}}
Als Jesus zum zweiten Mal vor dem Gericht die Stimme erhebt, ist es
tatsächlich nur zu diesem lapidaren \enquote{Du sagests}, das auch
von den Streichern nur mit drei knappen Akkorden gerahmt wird.

\subsubsection*{\Nr{61a.} \enquote{Eli, Eli, lama asabthani?}}
\enquote{Und um die neunte Stunde schrie Jesus laut}: Aus der Tiefe
ruft Jesus, aber laut; in b-moll ruft er, dessen extreme Eigenschaft
Mattheson sich gar nicht zu beschreiben anschickt, aber in hoher Lage;
verlassen von der Streicherbegleitung, als musikalische Realisierung
des \textit{status exinanitionis}, aber harmonisiert zeitweise mit
einem Stapel von vier dissonierenden Figuren. – Zutat Bach’s in der
Vertonung ist die Wiederholung des \enquote{lama?}, warum?; dies
zunächst als rhetorische Steigerung, aber auch, damit der Kunstgriff
der Übersetzung mit den gleichen Noten aufgeht.  Diese getreue
Übersetzung gibt der Evangelist dem Hörer, in seine Lage transponiert,
zugleich aber tonartlich noch tiefer herabgesunken in es-moll.

Zuletzt der Schrei Jesu in Nr.~61e. wird lediglich durch den Evangelisten
berichtet, mit einer klassischen exclamatio in der Sexte, und als
Dissonanz, die nur heteroleptisch durch die Orgel aufgelöst wird.

\section{Summa}

Wenn wir in dieser Form beobachten, wie Bach’s dramatische Konzeption
weniger von einer authentischen, modern-schauspielerischen Aktionsform
der Sänger, sondern von einer handwerklich gegründeten, \emph{mittelbaren}
\enquote{Expressivität} ausgeht, so ist das für die moderne Aufführungspraxis
sowohl ein Problem als auch eine Lösung.  Ein Problem deshalb, weil
weder die Gewohnheiten des Publikums noch die der Sänger mit einer
solchen Darstellungsweise rechnen – inzwischen kann man ja von
Aufführungen mit beispielsweise rekonstruierter barocker Gestik berichten,
die mit Sicherheit das Befremden beim Publikum erhöhen –; zugleich aber
eine Lösung, da eine authentische Darstellung Jesu immer als Extremfall
vor Augen führt, wie unrealistisch oder in diesem Fall eindeutig anmaßend
so ein Ideal doch ist.  Gerade bei einer Passion sind solche Überlegungen
notwendig.

Außerdem ist es hilfreich, auf diese Weise Bach ein wenig (weiter)
vom Sockel des \enquote{fünften Evangelisten} (Nathan
Söderblom\cite[S.~226]{thunberg}) herunterzuholen, und so letztlich einem
Verständnis näherzukommen.  Denn einerseits müssen wir damit umgehen,
dass Bach auch einem Umfeld entstammt, das unserem ziemlich fern ist,
und für ein völlig anderes Publikum schreibt; aber indem wir Bach’s
Sichtweise derart relativieren, sind wir zugleich herausgefordert, unser
eigenes Verhältnis zum Geschehen zu reflektieren, neu über Jesu Passion
nachzudenken, und so entfaltet sich eine neue Aktualität dieses Denkmals
unserer musikalischen Tradition.

\printbibliography

\end{document}
